\documentclass{mathbook}
\usepackage{setspace}
\usepackage{graphicx}
\usepackage{hyperref}
\usepackage{fontspec}
\usepackage{amsmath}
\usepackage[letterpaper, margin={0.75in}]{geometry}
\title{First Submission}
\author{Edouard Des Parois Perrault}
\date{\today}
\teacher{Mada Hoteit}
\assignment{Math Book First Submission}
\subtitle{A Symphony of Math}
\course{Grade 11 Enriched Mathematics IB A\&A Year 1}
\school{Lower Canada College}
\schoolLogo{images/LowerCanadaCollege.jpg}
\seal{images/edouard-seal-v1-01.png}
\setlength{\parindent}{0cm}
\setlength{\parskip}{0.5cm}
\setmainfont{CMU Sans Serif}
% \setlength{\lineskip}{2}
\begin{document}
    \maketitle
    \onehalfspacing
    % \linespread{1}
    \section{Summary}
    Du Sautoy begins his narrative by introducing us to a mathematician named David Hilbert. He draws our attention to a particular presentation Hilbert gave in August 1990; during the International Congress of Mathematicians in Paris that year, Hilbert challenged his audience with a set set of unsolved problems he believed should direct mathematical research for the century to come. Like the galaxy's farthest stars or the ocean's greatest depths, these problems presented the mathematicians of the time with a compelling mystery. Simply put, Hilbert gave his audience a math contest whose solutions would never be published by the CEMC. Although there was no official time limit and no material prize, the game was on, and all hoped solutions to the 23 problems would surface prior to the end of the century. \cite{2011} It may not come as much of a surprise that several of these problems remain unsolved today, given that they stumped many of the greatest mathematical minds of the time, though that's not to say that the excitement surrounding them has died down. Quite the contrary in fact, it has only increased over the years partly thanks to the Clay Mathematical Institute's efforts; in the year 2000, the Institute defined a smaller list, known as the millennium problems, with an incentive even greater than fame: fortune. A monetary prize of one million dollars was, and still is, promised to anyone who solves just one of these problems. \cite[p.~15]{Sautoy2003} Regardless of such a generous promise, all 23 problems have little significance in the context of novel, for Du Sautoy is particularly interested in one in particular, known as number eight on Herbert's original list, number two on Clay's list, or simply as the 'Riemann hypothesis'. \cite[p.~1]{Sautoy2003} This particular problem is a recurring motif throughout Du Sautoy's narrative, \footnote{Admittedly, this is not much of a surprise to the reader as it is mentioned on the back cover of the book} and is the glue that holds this incredibly intricate story together. It is the instrument through which Du Sautoy plays us the music of the primes. 

    Due to the fact that the Riemann hypothesis has remained unsolved for over a century, it's not very surprising that it's rather complicated \footnote{As is demonstrated by a great video by Numberphile \cite{Numberphile2014}}. Hence, Du sautoy refrains from explaining the theory in full at the beginning of his book. I like to see this novel as a police novel; the detective looks for clues and evidence in order to help him identify suspects, but remains unsuccessful at uncovering the true identity of the murderer until the very end. It's the suspense and the excitement of not knowing for sure who the culprit is that makes the narrative interesting. What fun would it be if the identity of the murderer was revealed at the very beginning? Answers to mathematical problems can be just as cunning as the serial killers that take on the role of villains in many of such books, for after a problem has been identified, its solutions avoid detection much like an expert thief; they hide in the shadows and travel only at night, such that their interception is nearly impossible. That said, in spite of these rather striking similarities, Du Sautoy's narrative differs from classic murder stories in one fundamental aspect, notwithstanding the absence of murder and death in a mathematical proof, of course; in the case of a murder mystery, the justification for solving the case is rather clear; crime is a social ill and must be prevented. Conversely, in the context of a mathematical mystery such as the Riemann Hypothesis, the justification for its pursuit isn't so lucid. Why don't I just read \emph{Murder on the Orient Express} \footnote{A great murder mystery narrative see \cite{Christie1934}} if I wanted a good murder story?
    
    Admittedly, although there may be some similarities between the two genres, \emph{The Music of the Primes} may not be the most ideal read for someone interested in murder and mystery. After all, Du Sautoy is no Agatha Christie. He does recognize, however, that the virtues of solving such a strange conjecture as the Riemann Hypothesis may not come immediately to the mind of the common mortal, and is this ability of his to explain in such a methodic and clear manner that makes \emph{The Music of the Primes} such an interesting read. At first, I feared that he may confine his justification to ``the Riemann Hypothesis seeks to understand the most fundamental objects in mathematics –– prime numbers'' \cite[p.~5]{Sautoy2003} Although this does answer the original question, it simply shifts the blame to another; why are \emph{primes} of any importance in the first place? Du Sautoy, however, has no intention to stop there, and consecrates several additional pages to responding to this second question. The answer, according to the author, is twofold. Some study primes simply by virtue of their beauty and their mystery. Indeed, prime numbers have both fascinated and frustrated with their inability to understand the underlying patterns of such a basic notion as an indivisible quantity. As Du Sautoy himself puts it, ``Randomness and chaos are anathema to the mathematician'' \cite[p.~7]{Sautoy2003} Once again, this passion for the aesthetic is not something that is shared by all. Austria, for instance, found prime numbers so useless that it converted tables of prime numbers into cartridges for its use in its war against Turkey. \cite[p.~47]{Sautoy2003} For those who fail to see beauty in numbers, DU Sautoy also brings up the usage of primes numbers in the world of commerce and security. The only way to determine for sure whether a number is prime is to divide it by all integers up to its square root. This is simple enough for a number such as 5 or 11, but gets rather complicated for larger numbers. It is based upon this idea that an algorithm, which is now referred to as RSA, was developed. RSA is heavily employed in cryptography, a discipline that solely concerns itself with securing information, whether this be computers or bank accounts. Communication over the web, for instance, is done through channels secured with RSA. If information is converted into a number, it can be multiplied by a set of large prime numbers, forming what is called a one-way function. The security of RSA stems from the fact that it takes an astronomically large amount of computational power to reverse a one-way function and to determine the original prime factors. \footnote{Given that RSA is one of the most secure forms of encryption currently available, I suppose it is evident that the algorithm is significantly more complex than this. See \cite{2017} and \cite{Lake2018} for more information on the implementation details of RSA.} One-way functions have several applications outside of securing information as well. Examples include hashing functions. \footnote{The functionality of hashing functions exceeds the scope of this assignment, but see \cite{2020} for more.} Although historically, primes have been studied simply because of their aesthetics, it is their complexity that makes them effective tools to outsmart devices capable of performing thousands of mathematical operations per second. As Du Sautoy himself puts it, ``The security of RSA depends on our ability to answer basic questions about prime numbers'' \cite[p.~12]{Sautoy2003} Evidently, there are several reasons that explain why prime numbers are so worthy of our time, and of an entire book named after them. Armed with these two motives, to understand primes for both their beauty and utility, Du Sautoy then ushers us into a vessel that subsequently takes us on a journey through the history of prime numbers, the mathematicians that worked on them, and the mathematical landscape that they came out of their research. Du Sautoy's vessel then blasts off into the mathematical sky with its final destination clearly labelled on the map laid on its bridge:~the Riemann Hypothesis
    
    Our first destination is Germany, where we encounter Gauß. There is a plethora of interesting characters that are mentioned throughout this book, but Gauß is by far one of the most fascinating members of the mathematical society, even today. Gauß gained considerable popularity when, in 1801, he published a series of calculations that predicted the position of the planetoid Ceres. \cite[p.~19]{Sautoy2003} Ceres had been recently discovered earlier that year by an astronomer named Giuseppe Piazzi, but its orbit had caused it to vanish shortly thereafter. This is a common occurrence with celestial objects, but it meant that mathematicians and astronomers had to subsequently engage themselves in a strange game of hide and seek; they needed to predict when and where it would appear next. In the end, Gauß emerged victorious, as prediction proved to be the closest. \cite{Weiss1999} Even as a young child, Gauß is known to have been incredibly precocious in the speed at which he grappled with complex mathematical concepts. For instance, there is a well-known story of how, when he was just 10, Gauß created a formula to find the \(n\)th triangular number, namely, \(\frac{1}{2} \cdot (N + 1) \cdot N\), impressing even his teacher.  \cite[p.~25]{Sautoy2003}

    Our next destination is not a country, but a topic; Du Sautoy proceeds to take us through an exploration of proof. He explains how the idea of mathematical proof, a concept that we take for granted today, has evolved over the course of the years. Although proof has its origins far earlier than the 17th century, Du Sautoy begins by introducing us to a man by the name of Pierre de Fermat. Today, Fermat is known as a talented mathematician, but back in his day, he was simply known as a judge. Math, to Fermat, was not an occupation, but a hobby. To him, and many other mathematicians of his time, proof wasn't so important. To be accepted as true, there was no need to show it was impossible for an idea to be false––a large amount of experimental evidence was enough to suggest it would always be true. In that respect, math in those days was much more similar to other sciences such as chemistry or physics; if the data suggests that it is true, it is considered true until the discovery of data that puts it in question. Fermat had little interest in numbers beyond his computational range. Hence, if a theorem was true so far as he could count, it was always true. This thought process may seem illogical, but makes a certain amount of sense if the context is taken into consideration. If something is true in every location it is applied, isn't that enough to say it is always true? If we took Fermat with us four centuries later, however, his understanding of proofs would doubtlessly have been dramatically shaken. In an age where space-travel seems within our reach, infinitely large and large numbers suddenly become of greater interest to us. Numbers outside of our computational reach seem closer than ever, and knowing about properties of these numbers without needed to calculate them becomes a huge asset. Unfortunately, this is purely hypothetical, for Fermat never got the chance to experience the 21st century. As a result, he didn't feel the need to prove many of his conjectures, which he simply scribbled in the margins of his copy of Diophantus's book \emph{Arithmetica}. In Fermat's time, there was no such thing as a blog. That said, if such communication channel had been available to him, I suspect it would have been filled with mathematical proofs and hypotheses, for Fermat was a rather prolifig writer 
    
    As mathematics progressed, however, it became clear that a mathematical conjecture could not be considered true so long as it was not proven to be the case. Large amounts of data may support a conjecture, but they do not prove it. Mathematicians saw proofs as a way to provide irrefutable evidence in favor of their theories. Du Sautoy shows us that Before Fermat's time, proving was very much a common practice among mathematicians. That said, in the early days of civilization, geometric proofs were common. This may partly be due to the fact that ancient mathematicians did not have access to algebra. Geometry was therefore an indispensable tool. Example of mathematicians who used these primitive techniques include Euclid. See \ref{proofs} for a reflection on this topic.

    After a pitstop in the village of Proofs and Conjectures, Du Sautoy ushers us back into his tour bus in order to take us to another different, yet connected, realm, that of Euler and of pre-revolutionary Europe. One of my favorite aspects of this novel is that it provides a much history context to the mathematical theories it elaborates. For instance, Du Sautoy talks about the monarchs in the 18th century who provided financial support to the great intellectuals of their time. Shortly before the French Revolution, monarchies such as Russia and France (ironically) were leaning more and more to the left. Instead of fostering incredibly conservative viewpoints, they embraced mathematics and science and sought to further its development. Catherine the Great, the empress if Russian, from 1762 to 1796, \cite{OldenbourgIdalie2020} had Euler in her court, along with painters, poets, and philosophers. \cite[pp.~41-42]{Sautoy2003} Although this summary is no place for a complete biography of Euler, he is known to have made several contributions to mathematics, from military to music. \footnote{See \cite{12tone2019} for a great video explaining Euler's theory on calculating the degree of `pleasantness' of a sound.}

    \section{Reflection}

    \subsection{A Refutation of Objective Mathematics}

    Is mathematics an objective reality? Is there something that exists at a more fundamental level? If there were no humans to count, would numbers and quantities still exist? This is very much an open question, and Du Sautoy seems to be of the opinion that there exists some from of objective reality. Math is math, and whether humans are there are not, two and two will make four. Du Sautoy quotes G.~H.~Hardy in writing that '317 is a prime not because we think so, or because our minds are shaped in one way or another, but \emph{because it is so}, because mathematical beauty is built that way.' Du Sautoy also quotes the French mathematician Alain Connes, ``There exists a mathematical reality outside of the human mind, a raw and immutable reality.'' Du Sautoy continues to say that ``at the heart of this world we find the unchaining list of primes.'' \cite[p.~7]{Sautoy2003}

    In many respects, answering this particular question is much like asking ``If a tree falls and no one hears it, does it make a sound?'' If there are two stones, but no one is there to count them, are there really two stones? I would argue that there is no reality outside of math, for math is defined by the human mind. Objects do not simply exist, they are brought into existence. Take the list of primes, for instance. What part of it does Du Sautoy consider immutable? The numerals? Certainly not, for there are many other ways to express the values 3 or 7, such as perhaps III or VII. The definition of what it means to be a prime? I would argue that this argument falls just as easily, as whether or not a number is prime is a \emph{definition} with which we have come up as humans. Chris K. Caldwell and Yeng Xiong from the Waterloo Department of Mathematics and Statistics state that ``whether or not a number (especially unity) is a prime is a matter of definition, so a matter of choice, context and tradition, not a matter of proof.'' \cite{Caldwell2012} We \emph{defined} a prime number as being a number greater than 1 that can only be divided by one and itself, but nothing stops us from expanding this definition. Of course, there are practical reasons for which 1 is not considered prime, but this does not prevent us from doing so. In his book, Du Sautoy alludes to Carl Sagan's novel \emph{contact}, where the main character receives a strange radio communication from space; a set of beeps. Two beeps, followed by three beeps, followed by five beeps, and so on, until the 100th prime number. The scientist assumes that only an intelligent life form would be able to produce such a specific sequence, and it is thanks to this intuition that the communication is correctly identified as originating from an extraterrestrial source. That said, couldn't the aliens produced a sequence of \emph{one} beep, followed by three beeps, and so on? No one ever said the definition of what it meant to be prime was a fundamental truth. Connes declares that math is a `universal language,' but how can it be so if the dictionary varies from one culture to another?

    Zero is also a rather special case. What does it mean for there to \emph{be} nothing? What does it mean to divide by nothing? This is a question that has confused and stumped generations of mathematicians. Some numeral systems didn't make use of zero at all. Why does $x^0=1$? Why does $0!=1$? Why does $\frac{x}{0}=\text{UNDEFINED}$? These are all matters of definition, and different cultures, and very likely different creatures throughout the universe, have a different understanding of them. Math is most definitely \emph{not} universal. That said, whether or not its existence presupposes the presence of a creature capable to count them is another story, one that is perhaps impossible to answer, since it would force us to cease to exist. How can we think if we do not exist? After all, \(x \cdot 0 = 0\)\dots

    \subsection{A Reflection on the Nature of Proofs}\label{proofs}

    Math and philosophy seem to be fundamentally related in many ways. One of the first ways we see such a connection is with proofs. In the book, Du Sautoy argues that ``proof in mathematics allows us to establish with 100 per cent certainty that facts about prime numbers will not change.'' \cite[pg.~32]{Sautoy2003} Du Sautoy says that proofs are statements so strong that they can never be disproven. The fundamental difference between math and other disciplines is that mathematical models can never become obsolete. Whereas in physics, the model of the universe has often been redesigned in order to correspond with more recent scientific discoveries, the mathematical model cannot become outdated, as its principles remain true. I beg to differ. Although it is undeniably true that math relies on something stronger than experimental evidence, that does not necessarily mean it cannot be beaten by time. The well-known mathematician Paul Erdős is famously quoted for having said that ``a problem worthy of attack proves its worth by fighting back.'' As time goes on, errors can be found in theorems, and perhaps Erdős is justified in saying that this is justification that they are worthy of being solved, but regardless, nothing is truly iron-clad. Surely, however, Du Sautoy has considered this himself, as he writes ``Of course, there is no guarantee that there isn't a subtle error, but ... we generally believe that errors can be spotted in proofs without waiting many years for new evidence.'' \cite[p.~33]{Sautoy2003} At this point, I am tempted to cite Andrew Wiles's proof of Fermat's last theorem, and how it was shown that he was incorrect after his publication of this proof, but in this situation, Du Sautoy's argument would still stand, since an error was found shortly after and was corrected. In other words, mathematical proofs are always correct since, if they have an error, this error can found directly on the page, and can therefore be addressed almost immediately. That said, we must remember, however, that as Socrates would say, ``I know that I know nothing.'' How can we ascertain for sure that mathematics is truly foolproof? It is this question that ties mathematics to one of the greatest inquiries of philosophy; what does it mean for something to be true? How can someone say that a claim they have is true? Are humans capable of achieving truth. Descartes addressed this extensively in his \emph{Meditations on First Philosophy}, \cite{Descartes1637} and believe that sensory, or empirical, information was not enough, for the existence of optic and sensory illusions suggested that it was possible for our senses to lie to us. There is no way to disprove, for instance, the belief that every time we count the studies of a triangle, we are deceived into believing that there are three sides, when there are in fact two, or four, or none, or some other quantity. \cite{Newman2019} As Descartes so eloquently states, ``How do I know \dots that there is no earth, no sky, no extended thing, no shape, no size, no place, while at the same time ensuring that all these things appear to me to exist just as they do now?'' All in all, in my opinion, whether or not a mathematical proof cannot be defeated is still very much an open question.

    It is with this 
    \newpage
    \bibliographystyle{IEEEtran}
    \bibliography{refs}
    %%% Extra %%%
    % Du Sautoy will return to Gauß later on in his book. In the meantime, we come across a mathematician which today has been given the name Fibonacci. Fibonacci is well known due to his discovery of the Fibonacci sequence in his investigation of rabbit reproduction patterns. In addition, the Fibonacci sequence shows up in all sorts of strange places, such as the position of petals in a flower \footnote{See \cite{Numberphile2018} for a great video on this topic.} or the ratio between someone's height to the distance between their feet and their bellybutton. 
    % Today, Gauß is considered to be one of the greatest mathematical minds that existed. Although he wasn't as productive as Euler, he made several contributions to a variety of fields throughout his lifetime. These contributions include modular arithmetic, which has a role to play in RSA encryption and modern computer science. These discoveries, which he published in his book \emph{Disquisitiones Arithmeticae}, are considered to comprise the foundations for a subject we call number theory today. \cite[p.~22]{Sautoy2003}
    %  That said, Du Sautoy recognizes that the importance of primes numbers is not evident to the reader right away I personally expected Du Sautoy to provide a deeply philosophical justification, along the lines of ``because math is beautiful'' or ``because it is human nature to solve problems'' or even just ``because,'' \footnote{That's not to say that these justifications are untrue or irrelevant, for they are not, but it is undeniably true that they are rather cliche. Take Henri Pointcarte's explanation, for example; ``The scientist does not study Nature because it is useful; he studies it because it is beautiful''\cite[p.~6]{Sautoy2003}} I was pleasantly surprised to find an alternative, more pragmatic, justification. The 
    % the unknown lies within the depths of the psyche of almost every human being. Perhaps the feeling that swept the audience as Hilbert presented his problems wasn't dissimilar from the sentiment we as students experience when we are presented with a blank math competition.
    % It is therefore no surprise that Du Sautoy refrains from explaining it at the beginning of his narrative. Instead, he goes on to talk about how the world once almost thought it was solved. He narrates a joke the mathematician Enrico Bomberi made for April fools that ended up being so successful even the United States government thought the hypothesis has been cracked. He told the world that the hypothesis had been solved, even if this was far from being the case. \par
    % \footnote{It seems to be the habit of mathematicians to produce overly cryptic justifications for their intentions. Take the polymath Pierre-Simon Laplace's praise of Euler, for instance. He says ``Read Euler, read Euler, he is the master of us all.'' Although we can guess that what he means to say is that he loves reading Euler, it's almost as though it takes a proof in itself to determine for sure what he is saying!}
\end{document}